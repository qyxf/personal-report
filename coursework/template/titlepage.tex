%标题页格式
\definecolor{MyLightBlue}{rgb}{0.8,1,1}

\let\origbaselineskip\baselineskip % 标题内部行距

% 无副标题的单双栏好了

\makeatletter

\begin{titlepage}
%	\pagecolor{MyLightBlue} % 背景颜色
	\begin{center}
		\begin{figure*}[!ht]\vspace{2em}
			\centering
			\includegraphics[width=0.5\textwidth]{xjtu}
		\end{figure*}

		\vspace{1em}	
		\heiti\erhao  课程作业 
		
		\vspace{2.3em}	
		\vspace{1em}		

		% 居中标题
		\begin{textblock*}{\textwidth}(1.15in+\hoffset,10cm) % {block width} (coords) 
			\centering
			\setlength{\baselineskip}{1.4\origbaselineskip} % 内部行距
			\heiti\chuhao \thetitle 
			
			
			\vspace{0.5em}
			\ifsubtitleexist
			\songti\yihao \the\subtitle
			\else
			\hrule 
			\vspace{0.5em}
			\pgfornamenthan[scale=0.4]{60}
%			\pgfornamenthan[scale=0.15]{58}
			\fi
		\end{textblock*}
		
		
		\vspace{13em}
		
		{\sanhao\yh\renewcommand{\arraystretch}{1.5}\renewcommand{\baselinestretch}{1.2}
		\begin{tabular}{c}
			姓名:\@author \\
			学号:\the\stuId \\
			班级:\the\class \\
			学院:\the\school \\
			\ifcsname @extrainfo\endcsname \@extrainfo\\ \fi
%			课程:\the\course\\ % 将来可做个选项,不仅可选,还要调整位置
			\@date \\
		\end{tabular}} 
			
		\renewcommand{\arraystretch}{1}
		{\global\let\author\@empty}%
		{\global\let\date\@empty}%
		\setcounter{footnote}{0}%
	\end{center}
	\begin{tikzpicture}[remember picture, overlay]
		\begin{pgfonlayer}{background}
		\node at ($(current page.south) +(0in,2.5in)$) {%
			\includegraphics[width=0.8\textwidth]{cover.png}};
		\end{pgfonlayer}
	\end{tikzpicture}
	\clearpage{\pagestyle{empty}\cleardoublepage}
\end{titlepage}
\makeatother

\newpage

\ifabstractexist
	\begin{center}
		\sanhao\selectfont\song
		摘\hspace{2em}要
	\end{center}
	\vspace{\baselineskip}
	\setcounter{page}{1}
	%\pagenumbering{Roman}
	\xiaosi
	
	\heiabstract \the\myabstract
	
	\vspace{\baselineskip}
	\heikeyword \the\mykeywords
	
	\vspace{\baselineskip}
	
	\begin{tikzpicture}[remember picture, overlay]
	\begin{pgfonlayer}{background}
	\node at ($(current page.south) +(0in,2.5in)$) {%
		\includegraphics[width=0.8\textwidth]{cover.png}};
	\end{pgfonlayer}
	\end{tikzpicture}
	
	\setcounter{footnote}{0}
	\setcounter{page}{0}
	\clearpage
\fi
