\usepackage{ctex}  
\usepackage{amsmath, amsfonts, amssymb} % 数学公式相关宏包
\usepackage{color}      % color content
\usepackage{graphicx}   % 导入图片
\usepackage{subfigure}  % 并排子图
\usepackage{hyperref}
\usepackage{url}
%\usepackage{breakurl}	%网址换行
\usepackage{bm}         % 加粗部分公式,比如\bm{aaa}aaa
\usepackage{multirow}
\usepackage{booktabs}
\usepackage{longtable}  %长表格
\usepackage{supertabular}%跨页表格
\usepackage{changepage}
\usepackage{siunitx}%输入角度
\usepackage{appendix}
\newcommand{\mycomment}[1]{}%自定义注释。。

%修改字体,不会就去看刘海洋的书,不要自己瞎鼓捣
%ctex有的就直接用命令字体,可以不用这样调用\setCJKfamilyfont。当然renew后方便些
\usepackage{fontspec, xunicode, xltxtra}
\usepackage{xeCJK}		%中文字体
%\setCJKfamilyfont{song}{SimSun}         %宋体 song,\songti
\newcommand{\song}{\CJKfamily{song}}                        
%\setCJKfamilyfont{fs}{FangSong_GB2312} %仿宋2312 fs,\fangsong
\newcommand{\fs}{\CJKfamily{fs}}                            
%\setCJKfamilyfont{yh}{Microsoft YaHei}  %微软雅黑 yh,\yahei 
\newcommand{\yh}{\CJKfamily{yh}}  
%\setCJKfamilyfont{hei}{SimHei}          %黑体  hei,\heiti 
\newcommand{\hei}{\CJKfamily{hei}}    
%\setCJKfamilyfont{hwxh}{STXihei}        %华文细黑  hwxh  
\newcommand{\hwxh}{\CJKfamily{hwxh}}
%\setCJKfamilyfont{asong}{Adobe Song Std}%Adobe 宋体  asong  
\newcommand{\asong}{\CJKfamily{asong}}
%\setCJKfamilyfont{ahei}{Adobe Heiti Std}%Adobe 黑体  ahei  
\newcommand{\ahei}{\CJKfamily{ahei}}  
%\setCJKfamilyfont{akai}{Adobe Kaiti Std}%Adobe 楷体  akai,\kaishu  
\newcommand{\akai}{\CJKfamily{akai}}
%\setCJKfamilyfont{hwsong}{STSong}                            %华文宋体  hwsong
\newcommand{\hwsong}{\CJKfamily{hwsong}}
%\setCJKfamilyfont{hwzs}{STZhongsong}                        %华文中宋  hwzs
\newcommand{\hwzs}{\CJKfamily{hwzs}}
%\setCJKfamilyfont{hwfs}{STFangsong}                            %华文仿宋  hwfs
\newcommand{\hwfs}{\CJKfamily{hwfs}}
%\setCJKfamilyfont{hwl}{STLiti}                                        %华文隶书  hwl
\newcommand{\hwl}{\CJKfamily{hwl}}
%\setCJKfamilyfont{hwxw}{STXinwei}                                %华文新魏  hwxw
\newcommand{\hwxw}{\CJKfamily{hwxw}}
%\setCJKfamilyfont{hwk}{STKaiti}                                    %华文楷体  hwk
\newcommand{\hwk}{\CJKfamily{hwk}}
%\setCJKfamilyfont{hwxk}{STXingkai}                            %华文行楷  hwxk
\newcommand{\hwxk}{\CJKfamily{hwxk}}
%\setCJKfamilyfont{hwcy}{STCaiyun}                                 %华文彩云 hwcy
\newcommand{\hwcy}{\CJKfamily{hwcy}}
%\setCJKfamilyfont{hwhp}{STHupo}                                 %华文琥珀   hwhp
\newcommand{\hwhp}{\CJKfamily{hwhp}}
%\setCJKfamilyfont{fzsong}{Simsun (Founder Extended)}     %方正宋体超大字符集   fzsong
\newcommand{\fzsong}{\CJKfamily{fzsong}}
%\setCJKfamilyfont{fzyao}{FZYaoTi}                                    %方正姚体  fzy
\newcommand{\fzyao}{\CJKfamily{fzyao}}
%\setCJKfamilyfont{fzshu}{FZShuTi}                                    %方正舒体 fzshu
\newcommand{\fzshu}{\CJKfamily{fzshu}}


\setCJKsansfont{\hei}%serif是有衬线字体sans serif无衬线字体。
%\setCJKmonofont{\fs}%mono字体
%\setCJKmainfont{Microsoft YaHei}  % 微软雅黑
%\setCJKmainfont{YouYuan}  % 幼圆,没有
%\setCJKmainfont{LiShu}  % 隶书
%\setCJKmainfont{NSimSun}  % 新宋体
%\setCJKmainfont{KaiTi}    % 楷体
%\setCJKmainfont{SimSun}   % 宋体
%\setCJKmainfont{\hei}   % 黑体
%\setCJKmainfont{Segoe UI} %GitHub主题,不认中文

% -- 英文字体 --
\setmainfont{Times New Roman}
%\setmainfont{Consolas}
\setsansfont{Verdana}
%\setsansfont{DejaVu Sans}
%\setmonofont{Latin Modern Mono}

% 其他
%\usepackage{ccfonts} %公式使用concrete系列

% -- 字号设置 --
% ctex有设置好的中文字号,不过这是自定义的方法
\newcommand{\chuhao}{\fontsize{42pt}{\baselineskip}\selectfont}
\newcommand{\xiaochu}{\fontsize{36pt}{\baselineskip}\selectfont}
\newcommand{\yihao}{\fontsize{28pt}{\baselineskip}\selectfont}
\newcommand{\xiaoyi}{\fontsize{24pt}{\baselineskip}\selectfont}
\newcommand{\erhao}{\fontsize{21pt}{\baselineskip}\selectfont}
\newcommand{\xiaoer}{\fontsize{18pt}{\baselineskip}\selectfont}
\newcommand{\sanhao}{\fontsize{15.75pt}{\baselineskip}\selectfont}
\newcommand{\sihao}{\fontsize{14pt}{\baselineskip}\selectfont}
\newcommand{\xiaosi}{\fontsize{12pt}{\baselineskip}\selectfont}     
\newcommand{\wuhao}{\fontsize{10.5pt}{\baselineskip}\selectfont}
\newcommand{\xiaowu}{\fontsize{9pt}{\baselineskip}\selectfont}
\newcommand{\liuhao}{\fontsize{7.875pt}{\baselineskip}\selectfont}

%修改摘要
\usepackage{abstract}
\newcommand{\heiabstract}{\noindent{\xiaosi\selectfont\hei 摘\hspace{0.5em}要:}}
\newcommand{\heikeyword}{\noindent{\xiaosi\selectfont\hei 关\hspace{0.5em}键\hspace{0.5em}词:}}
\newcommand{\heitypeofthesis}{\noindent{\xiaosi\selectfont\hei 论文类型:}}
\newtoks\myabstract%摘要
\newtoks\mykeywords%关键词

%修改章节标题的字体,ctexart才能用
%用了ctexart,还使章节标题居中
%\CTEXsetup[format={\hei}]{section}
%\CTEXsetup[format={\hei}]{subsection}
%\CTEXsetup[format={\hei}]{subsubsection}
%修改章节标题为中文
\usepackage{zhnumber}
%\renewcommand\thesection{\zhnum{section}}
%\renewcommand \thesubsection {\arabic{subsection}}
%\renewcommand \thesubsubsection {\arabic{subsection}.\arabic{subsubsection}}


%参考文献
\usepackage{natbib}
\usepackage[super]{gbt7714}%authoryear
%用选项代替\bibliographystyle{gbt7714-plain}unsrt
%\newcommand{\heirefname}{\hei 参考文献:\song}
\renewcommand{\abstractname}{}%改摘要的格式
\newcommand{\upcite}[1]{$^{\mbox{\scriptsize \cite{#1}}}$}%参考文献的上标引用

%一些信息
\newtoks\stuId%学号
\newtoks\class%班级
\newtoks\school%学院
\newtoks\course%课程名
\newtoks\partner%同组成员

% 定义页眉和页脚
\usepackage{fancyhdr}
\pagestyle{fancy}
\usepackage{pgfornament-han,pgfornament}
\newcommand{\footdeco}{
	\resizebox{10em}{5ex}{
		\begin{tikzpicture}
		\node at (0,0) {\pgfornamenthan[scale=0.05]{58}};
		\node[opacity=0.2, draw opacity=0] at (0.02,-0.03) {\pgfornamenthan[scale=0.05]{58}};
		\end{tikzpicture}}\\[-1ex]}
\newcommand{\makeheadrule}{
	%\rule[6pt]{\textwidth}{0.5pt} \\[-17.5pt]
	\rule{\textwidth}{0.5pt}}
\renewcommand{\headrule}{
	{\if@fancyplain\let\headrulewidth\plainheadrulewidth\fi
		\makeheadrule}}
%\newcommand{\thetitle}{\makeatletter\@title\makeatother}
%\lhead{\heiti 这里还没弄好}%\thetitle
%\fancyhead[R]{第 \bfseries\thepage 页}
\fancyfoot[C]{\footdeco \heiti 西安交通大学课程作业}

%tikz
\usepackage{tikz}
\usetikzlibrary{calc,backgrounds}
\tikzset{
	arrow1/.style = {
		draw = black, thick, -{Latex[length = 4mm, width = 1.5mm]},
	}
}

% 汉风图纹	
\RequirePackage{pgfornament-han}
\RequirePackage{pgfornament}


%修改列表环境
\usepackage{enumitem}
\setenumerate[1]{itemsep=0pt,partopsep=0pt,parsep=\parskip,topsep=5pt,leftmargin=2em}
\setitemize[1]{itemsep=0pt,partopsep=0pt,parsep=\parskip,topsep=5pt}
\setdescription{itemsep=0pt,partopsep=0pt,parsep=\parskip,topsep=5pt}

%彩色文本框
\usepackage{tcolorbox}
\tcbuselibrary{skins, breakable, theorems}
\usepackage{colortbl}
\tcbset{colframe = blue!50!black, colback = white,
	colupper = red!50!black, fonttitle = \bfseries,
	nobeforeafter, center title}
%更改脚注格式
\usepackage{color}
\renewcommand{\thefootnote}{\color{red}\arabic{footnote}}

%插入代码
\usepackage{listings}
\lstset{%自己定制的MATLAB格式,用mcode字体为啥不太好?
	backgroundcolor=\color{red!5!green!5!blue!0},%代码块背景色为浅灰色
	rulesepcolor= \color{gray}, %代码块边框颜色
	breaklines=true,  %代码过长则换行
	numbers=left, %设置行号位置
	numberstyle=\scriptsize, %设置行号大小
	basicstyle=\ttfamily,% 基本代码风格
	keywordstyle=\color{blue}, %设置关键字颜色\bfseries
	commentstyle=\rmfamily\color[cmyk]{1,0,1,0}, %设置注释颜色
	stringstyle=\ttfamily\color{red!60!green!40!blue!80},
	frame=shadowbox,%用方框框住代码块
	%frame=singl
	escapeinside=``, %逃逸字e, %设置边框格式符(1左面的键),用于显示中文
	extendedchars=false, %解决代码跨页时,章节标题,页眉等汉字不显示的问题
	xleftmargin=2em,xrightmargin=2em, aboveskip=1em, %设置边距
	tabsize=4, %设置tab空格数
	showspaces=false, %不显示空格
	showstringspaces=false,
	captionpos=t,% 这段代码的名字所呈现的位置,t指的是top上面
	columns=fullflexible
}
%\usepackage[framed,numbered,useliterate]{mcode}
%basicstyle= \sffamily,% 基本代码风格
%stringstyle= \ttfamily,
%commentstyle=\rmfamily\itshape,%注释的风格,斜体

%罗马数字
\newcommand{\RNum}[1]{\uppercase\expandafter{\romannumeral #1\relax}}

%写化学式
\usepackage[version=4]{mhchem}
